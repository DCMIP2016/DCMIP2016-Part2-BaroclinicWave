%% Copernicus Publications Manuscript Preparation Template for LaTeX Submissions
%% ---------------------------------
%% This template should be used for copernicus.cls
%% The class file and some style files are bundled in the Copernicus Latex Package which can be downloaded from the different journal webpages.
%% For further assistance please contact the Copernicus Publications at: publications@copernicus.org
%% http://publications.copernicus.org


%% Please use the following documentclass and Journal Abbreviations for Discussion Papers and Final Revised Papers.


%% 2-Column Papers and Discussion Papers
\documentclass[gmd, manuscript]{copernicus}



%% Journal Abbreviations (Please use the same for Discussion Papers and Final Revised Papers)

% Archives Animal Breeding (aab)
% Atmospheric Chemistry and Physics (acp)
% Advances in Geosciences (adgeo)
% Advances in Statistical Climatology, Meteorology and Oceanography (ascmo)
% Annales Geophysicae (angeo)
% ASTRA Proceedings (ap)
% Atmospheric Measurement Techniques (amt)
% Advances in Radio Science (ars)
% Advances in Science and Research (asr)
% Biogeosciences (bg)
% Climate of the Past (cp)
% Drinking Water Engineering and Science (dwes)
% Earth System Dynamics (esd)
% Earth Surface Dynamics (esurf)
% Earth System Science Data (essd)
% Fossil Record (fr)
% Geographica Helvetica (gh)
% Geoscientific Instrumentation, Methods and Data Systems (gi)
% Geoscientific Model Development (gmd)
% Geothermal Energy Science (gtes)
% Hydrology and Earth System Sciences (hess)
% History of Geo- and Space Sciences (hgss)
% Journal of Sensors and Sensor Systems (jsss)
% Mechanical Sciences (ms)
% Natural Hazards and Earth System Sciences (nhess)
% Nonlinear Processes in Geophysics (npg)
% Ocean Science (os)
% Proceedings of the International Association of Hydrological Sciences (piahs)
% Primate Biology (pb)
% Scientific Drilling (sd)
% SOIL (soil)
% Solid Earth (se)
% The Cryosphere (tc)
% Web Ecology (we)
% Wind Energy Science (wes)


%% \usepackage commands included in the copernicus.cls:
%\usepackage[german, english]{babel}
%\usepackage{tabularx}
%\usepackage{cancel}
%\usepackage{multirow}
%\usepackage{supertabular}
%\usepackage{algorithmic}
%\usepackage{algorithm}
%\usepackage{amsthm}
%\usepackage{float}
%\usepackage{subfig}
%\usepackage{rotating}

% Custom commands
\newcommand{\vb}{\mathbf}
\newcommand{\vg}{\boldsymbol}
\newcommand{\mat}{\mathsf}
\newcommand{\diff}[2]{\frac{d #1}{d #2}}
\newcommand{\diffsq}[2]{\frac{d^2 #1}{{d #2}^2}}
\newcommand{\pdiff}[2]{\frac{\partial #1}{\partial #2}}
\newcommand{\pdiffsq}[2]{\frac{\partial^2 #1}{{\partial #2}^2}}


\begin{document}

\title{DCMIP2016, Part 2: Moist Baroclinic Wave}


% \Author[affil]{given_name}{surname}

\Author[1]{Christiane}{Jablonowski}
\Author[2]{Peter H.}{Lauritzen}
\Author[3]{James}{Kent}
\Author[2]{Ramachandran}{Nair}
\Author[4]{Kevin A.}{Reed}
\Author[5]{Paul A.}{Ullrich}
\Author[2]{Colin M.}{Zarzycki}

\Author[6]{David A.}{Hall}

\Author[7]{Alex}{Reinecke}
\Author[7]{Kevin}{Viner}

\Author[8]{Don}{Dazlich}
\Author[8]{Ross}{Heikes}
\Author[8]{Celal}{Konor}
\Author[8]{David}{Randall}

\Author[9]{Xi}{Chen}
\Author[9]{Lucas}{Harris}

\Author[10]{Marco}{Giorgetta}
\Author[11]{Daniel}{Reinert}

\Author[12]{Christian}{Kuehnlein}

\Author[13]{Robert}{Walko}

\Author[14]{Jean}{de Grandpr\'e}
\Author[14]{Vivian}{Lee}
\Author[14]{Abdessamad}{Qaddouri}

\Author[15]{Hiroaki}{Miura}
\Author[15]{Tomoki}{Ohno}
\Author[16]{Ryuji}{Yoshida}

\Author[2]{Joseph}{Klemp}
\Author[2]{Sang-Hun}{Park}
\Author[2]{William}{Skamarock}

\Author[17]{Thomas}{Dubos}
\Author[17]{Yann}{Meurdesoif}

\Author[18]{Elijah}{Goodfriend}
\Author[18]{Hans}{Johansen}

\affil[1]{University of Michigan}
\affil[2]{National Center for Atmospheric Research}
\affil[3]{University of South Wales}
\affil[4]{Stony Brook University}
\affil[5]{University of California, Davis}
\affil[6]{University of Colorado, Boulder}
\affil[7]{Naval Research Laboratory}
\affil[8]{Colorado State University}
\affil[9]{Geophysical Fluid Dynamics Laboratory}
\affil[10]{Max Planck Institute for Meteorology}
\affil[11]{Deutscher Wetterdienst (DWD)}
\affil[12]{European Center for Medium-Range Weather Forecasting}
\affil[13]{University of Miami}
\affil[14]{Environment Canada}
\affil[15]{University of Tokyo}
\affil[16]{RIKEN}
\affil[17]{Institut Pierre-Simon Laplace (IPSL)}
\affil[18]{Lawrence Berkeley National Laboratory}

%% The [] brackets identify the author with the corresponding affiliation. 1, 2, 3, etc. should be inserted.



\runningtitle{DCMIP2016: Moist Baroclinic Wave}

\runningauthor{Jablonowski, et al.}

\correspondence{Christiane Jablonowski (cjablono@umich.edu)}



\received{}
\pubdiscuss{} %% only important for two-stage journals
\revised{}
\accepted{}
\published{}

%% These dates will be inserted by Copernicus Publications during the typesetting process.


\firstpage{1}

\maketitle



\begin{abstract}
This paper discusses a new idealized test for atmospheric dynamical cores.
\end{abstract}



\introduction  %% \introduction[modified heading if necessary]

The baroclinic instability test of \cite{ullrich2014proposed} considers a reference state in geostrophic and hydrostatic balance that satisfies the conditions for baroclinic instability.  Although a perfect model should be able to maintain this state indefinitely, small truncation errors associated with numerical inaccuracies and grid structure will trigger the development of the wave modes associated with baroclinic development.  To control the development of the baroclinic wave, a small perturbation (but one which is large compared with machine truncation) is added to the flow so as to trigger the development of a wave over a period of approximately 10 days.  A moist variant of the dry dynamical test of \cite{ullrich2014proposed} is considered here so as to understand the impact of moisture feedbacks on the development of the wave.

This test case is similar in character to the test of \cite{jablonowski2006baroclinic}, but has a number of key differences:  (1) this test is an analytical solution of the equations of motion in height ($z$) coordinates, (2) the bottom topography is zero throughout the domain, (3) the new test case does not have a distinct stratosphere (the presence of a stratosphere is largely irrelevant for understanding baroclinic development), and (4) the velocity field goes to zero at the model surface.
 
\begin{table}[h]

\caption{List of constants used for the Moist Baroclinic Wave test case}
\label{test4:tab}
\begin{tabular*}{\textwidth}{@{\extracolsep{\fill}}lll}
\hline Constant & Value & Description \\
\hline 
$z_{\tiny \mbox{top}}$ & $44000\ \mbox{m}$ & Recommended height position of the model top \\
$p_{\tiny \mbox{top}}$ & $\approx 2.26$ hPa & Recommended pressure at the model top\\
$X$ & $1$ & Reduced-size planet scaling factor, see below\\
$a$ & $a_{\tiny \mbox{ref}}/X$ & Scaled radius of the Earth \\
$\Omega$ & $\Omega_{\tiny \mbox{ref}}X$ & Scaled angular speed of the Earth \\
$p_s$ & $1000\ \mbox{hPa}$ & Surface pressure (constant) \\
$p_0$ & $1000\ \mbox{hPa}$ & Reference pressure (constant) \\
$u_0$ & $35\ \mbox{m\ s}^{-1}$ & Maximum amplitude of the zonal wind \\
$b$ & $2$ & Half-width parameter \\
$K$ & $3$ & Power used for temperature field \\
$T_E$ & $310\ \mbox{K}$ & Horizontal-mean temperature at the surface \\
$T_P$ & $240 \ \mbox{K}$ & Temperature at the polar surface\\
$u_p$ & $1\ \mbox{m\ s}^{-1}$ & Maximum amplitude of the zonal wind perturbation \\
$z_p$ & $15000\ \mbox{m}$ & Maximum height of the zonal wind perturbation \\
$\lambda_p$ & $\pi / 9$ & Longitude of the zonal wind perturbation centerpoint (20$^\circ$ E)\\
$\varphi_p$ & $2 \pi / 9$ & Latitude of the zonal wind perturbation centerpoint (40$^\circ$ N)\\
$R_p$ & $a / 10$ & Radius of the zonal wind perturbation \\
%$T_E$ & $310 \ \mbox{K}$ & Temperature at the equator\\
$\Gamma$ & $0.005\ \mbox{K\ m}^{-1}$ & Temperature lapse rate \\
$\Delta T$ & $4.8 \times 10^{5}\ \mbox{K}$ & Empirical temperature difference \\
$\varphi_w$ & $2 \pi / 9$ & Specific humidity latitudinal width parameter $(40^\circ)$\\
$p_w$ & $340\ \mbox{hPa}$ & Specific humidity vertical pressure width parameter \\
$q_0$ & $0.018$ kg/kg& Maximum specific humidity amplitude \\
$q_t$ & $1.0 \times 10^{-12}$ kg/kg & Specific humidity above artificial tropopause \\
$p_t$ & $10000\ \mbox{hPa}$ & Pressure at artificial tropopause \\  
\hline 
\end{tabular*}

\end{table}

\subsection{Reference State}

This section describes the analytical form of the reference state for the baroclinic wave.  The test case is initialized with a constant surface pressure and with a surface geopotential equal to zero.  The meridional wind in the reference state is zero.

In the reference state, the virtual temperature is given by
\begin{equation}
T_v(\varphi, z) = \frac{1}{\tau_1(z)-\tau_2(z) I_T(\varphi)},
\label{virtTemp}
\end{equation} where $I_T(\varphi)$ is defined as
\begin{equation}
I_{T}(\varphi) =(\cos \varphi )^K-\frac{K}{K+2}(\cos \varphi )^{K+2},
\end{equation} and $\tau_1(z)$ and $\tau_2(z)$ are defined as follows:
\begin{align}
\tau_1(z) &= \frac{1}{T_0} \exp\left(\frac{\Gamma z}{T_0}\right) + \left( \frac{T_0-T_P}{T_0T_P} \right)\left[1-2\left(\frac{z g}{b R_d T_0}\right)^2\right] \exp\left[-\left(\frac{z g}{b R_d T_0}\right)^2\right] \\
\tau_2(z) &= \frac{(K + 2)}{2} \left( \frac{T_E-T_P}{T_E T_P} \right) \left[1-2\left(\frac{z g}{b R_d T_0}\right)^2\right] \exp\left[-\left(\frac{z g}{b R_d T_0}\right)^2\right],
\end{align} with $T_0 = \tfrac{1}{2} (T_E + T_P)$.  To maintain hydrostatic balance, the pressure is given by:
\begin{equation}
p(\varphi, z) = p_0\exp \left[ -\frac{g}{R_d}(\tau_{\text{int},1}(z) -\tau_{\text{int},2}(z) I_T(\varphi) ) \right]
\end{equation} with $\tau_{\text{int},1}(z)$ and $\tau_{\text{int},2}(z)$ given by
\begin{align}
\tau_{\text{int},1}(z) &=\frac{1}{\Gamma} \left[ \exp\left( \frac{\Gamma z}{T_0} \right)-1 \right] + z \left(\frac{T_0-T_P}{T_0T_P} \right) \exp\left[-\left(\frac{z g}{b R_d T_0}\right)^2\right] \\
\tau_{\text{int},2}(z) &=\frac{(K+2)}{2} \left(\frac{T_E-T_P}{T_E T_P} \right) z \exp\left[-\left(\frac{z g}{b R_d T_0}\right)^2\right].
\end{align}  If density is a prognostic variable, it can be obtained from $p$, $T_v$ and the ideal gas law (\ref{eq:idealgaslaw}).  Finally, the zonal velocity is
\begin{equation}
u_{\text{ref}}(\varphi, z) = -\Omega_{\text{ref}} a_{\text{ref}} \cos(\varphi)+\sqrt{(\Omega_{\text{ref}} a_{\text{ref}} \cos(\varphi))^2+ a_{\text{ref}} \cos(\varphi)U(z,\varphi))},
\end{equation} where
\begin{equation}
U(z, \varphi) = \frac{g K}{a_{\text{ref}}} \tau_{\text{int}_2}(z) \left[ (\cos \varphi)^{K - 1} - (\cos \varphi)^{K + 1} \right] T_v(\varphi, z).
\end{equation} 

\subsection{Perturbations}

To trigger the development of the baroclinic wave, a perturbation is applied to the zonal velocity field that takes the form of a simple exponential bell with a vertical taper:
\begin{equation}
u^\prime(\lambda, \varphi, z) = \left\{ \begin{array}{ll} \displaystyle u_p Z_p(z)  \exp \left[ - \left( \frac{R(\lambda, \varphi; \lambda_p, \varphi_p)}{R_p} \right)^2 \right], & \mbox{if $R(\lambda, \varphi; \lambda_p, \varphi_p) < R_p$,} \\ 0, & \mbox{otherwise,} \end{array} \right.
\end{equation} where
\begin{equation}
Z_p(z) = \left\{ \begin{array}{ll} \displaystyle 1 - 3 \left( \frac{z}{z_p} \right)^2 + 2 \left( \frac{z}{z_p} \right)^3, & \mbox{if $z \leq z_p$,} \\ 0, & \mbox{otherwise.} \end{array} \right.
\end{equation}  Consequently, the perturbed velocity field takes the form
\begin{equation}
u(\lambda, \varphi, z) = u_{\text{ref}}(\varphi, z) + u^\prime(\lambda, \varphi, z).
\end{equation}

\subsection{Moist initial conditions}

We define the vertical $\eta$ coordinate as
\begin{equation}
\eta(\lambda, \varphi, z) = p(\lambda, \varphi, z) / p_s.
\end{equation}  Since the surface pressure of the moist air $p_s$ is constant with $p_s = p_0 = 1000$ hPa  the vertical coordinate $\eta$ is represented by $\eta = p/p_0$.  Specific humidity is specified in terms of $\eta$ as
\begin{eqnarray}
q(\lambda, \varphi, \eta) &=& \left\{ \begin{array}{ll} \displaystyle q_0 \exp\Bigg[- \Big(\frac{\varphi}{\varphi_{w}}\Big)^4 \Bigg] \exp\Bigg[- \Bigg(\frac{(\eta-1)p_0}{p_{w}}\Bigg)^2  \Bigg], & \mbox{if $\eta > p_t / p_s$,} \\ q_{t}, & \mbox{otherwise.} \end{array} \right.
\end{eqnarray} The functional form of $q$ and its parameters were inspired by observations. This moisture fields leads to maximum relative humidities around 85\% in the lower levels of the midlatitudes.

Note that the moist temperature is colder than the temperature one would obtain with $q = 0$. However, note that in the moist case the virtual temperature and moist pressure determine the strength of the pressure gradient term in the momentum equations. Since these are identical to the temperature and pressure in the dry case, the forcing by the pressure gradient term is the same in both the dry and moist variant of the baroclinic wave. The moist variant of the baroclinic wave without the temperature forcing from large-scale condensation should lead to almost identical results when compared to the dry version. Very small variations are expected since the moisture gets independently transported as a passive tracer in this case and some models utilize the moist variant of the physical constant $c_p$. If possible, the dry $c_p$ should be used. Comparing the evolution of the dry baroclinic wave to its moist variant (without large-scale condensation) can serve as a first sensibility check.

\begin{figure}[tb]
\center\includegraphics[width=\linewidth]{plot_baroclinicwave_init.pdf}
  \caption{Initial state for the moist baroclinic wave test.}\label{fig:baroclinicwave_init}
\end{figure} 

%
%
%
%
%
\subsection{Terminator `toy'-chemistry}
The terminator `toy'-chemistry is presented in \cite{lauritzen2015terminator} and mimics photolysis-driven processes near the solar terminator. Two passive{\footnote{i.e. tracers do not feed back on the flow}} tracers, Cl and Cl$_2$, that are chemically reactive are transported. The sources and sinks are given by a simple, but non-linear, `toy' chemistry. As a result, strong gradients in the spatial distribution of the species develop near the edge of the terminator. Despite the large spatial variations in Cl and Cl$_2$ the weighted sum Cl$_y$=Cl+2Cl$_2$ should always be preserved in any flow field (if the initial condition for Cl$_y$ is constant). An overview of the `toy' terminator chemistry is given in Appendix \ref{sec:ToyChemistry}. The terminator test demonstrates how well the advection/transport scheme and/or physics-dynamics coupling preserves linear correlations.

The initial conditions for Cl and Cl$_2$ (in terms of {\bf{dry}} mixing ratios $q_{Cl}$ and $q_{Cl2}$, respectively) are the steady-state solutions to the terminator chemistry with no flow \cite{lauritzen2015terminator} (see Figure \ref{fig:terminator-2d-se-day0}):
\begin{figure}[t]
\includegraphics[width=\linewidth]{terminator-2d-se-day0.pdf}
\caption{Contour lines of $<q_{Cl}>/(4.0\times 10^{-6})$, (upper left), $<q_{Cl2}>/(4.0\times 10^{-6})$ (upper right), $<q_{Cly}>$ (lower left) and surface pressure (lower right) at day 0 (initial conditions).}
\label{fig:terminator-2d-se-day0}
\end{figure}
\begin{eqnarray}
q_{Cl}(\lambda,\theta,z,t=0)&=&D-r, \\
q_{Cl2}(\lambda,\theta,z,t=0)&=&\frac{1}{2}\left( q_{Cly} - D + r\right),
\end{eqnarray}
where $q_{Cly}=4.0\times 10^{-6}$kg/kg,
\begin{align}
 r &= \frac{k_1}{4 k_2}, \\
 D &= \sqrt{r^2 + 2 r q_{Cly}},
\end{align}
and reaction coefficients $k_1$ and $k_2$ are given in appendix \ref{sec:ToyChemistry} (equations \eqref{app:term_k1} and \eqref{app:term_k2}). Please note that the mixing ratios are dry, i.e. the ratio between the density of the species and the density of dry air. 

The forcing terms are computed analytically (assuming no flow) over a (physics/chemistry) time-step $\Delta t$:
\begin{equation}
F_{Cl}^n= - L_{\Delta t} \frac{ (q_{Cl}^n - D + r ) (  {q_{Cl}}^n  + D + r  ) }{ 1+e^{-4 k_2 D \Delta t} + \Delta t L_{\Delta t} (q_{Cl}^n + r )  }.
\end{equation}
where $q_{Cl}^n$ is the value of $q_{Cl}$ at the beginning of the n'th time step and
\begin{equation}
  L_{\Delta t} = 
  \left\{ 
     \begin{array}{cc}  
          \frac{1-e^{-4 k_2 D \Delta t}}{D \Delta t}  &\mbox{if }  D > 0 \\[1ex]
          4k_2 &\mbox{if } D = 0.
      \end{array} 
   \right. \label{eq:linvsexp}
\end{equation}
and by conservation,
\begin{equation}
F_{\mathrm{Cl}_2  }^n = -\frac{1}{2} F_{\mathrm{Cl}  }^n .
\end{equation}
In implementation, $L_{\Delta t} $ needs some care.  As $4 k_2 D \Delta t$ approaches machine precision, it is useful to simply use the formula for $D=0$ rather than the expression for $D>0$. The chemistry/physics updated mixing ratios are given by
\begin{eqnarray}
q^n_{Cl}+\Delta t\, F_{Cl}^n,\\
q^n_{Cl2}+\Delta t\, F_{Cl2}^n.
\end{eqnarray}
In terms of Fortran code the analytical forcing is given by:
\begin{verbatim}
  ! dt is size of physics time step
  cly = cl + 2.0*cl2

  r = k1 / (4.0*k2)
  d = sqrt( r*r + 2.0*r*cly )
  e = exp( -4.0*k2*d*dt )

  if( abs(d*k2*dt) .gt. 1e-16 )
    el = (1.0-e) / (d*dt)
  else
    el = 4.0*k2
  endif

  f_cl  = -el * (cl-d+r) * (cl+d+r) / (1.0 + e + dt*el*(cl+r))
  f_cl2 = -f_cl / 2.0
\end{verbatim}
The reaction rates are defined by
\begin{verbatim}
  ! k1 and k2 are reaction rates
  k1_lat_center =  20.0 ! degrees
  k1_lon_center = 300.0 ! degrees
  k1 = max(0.d0,sin(lat)*sin(k1_lat_center) 
           + cos(lat)*cos(k1_lat_center)*cos(lon-k1_lon_center))
  k2 = 1.0
\end{verbatim}
The initial condition is defined by
\begin{verbatim}
  cly = 4.0e-6

  r = k1 / (4.0*k2)
  d = sqrt( r*r + 2.0*cly*r )

  cl = d-r
  cl2 = cly / 2.0 - (d-r) / 2.0
\end{verbatim}
Fortran subroutines that, given $(\lambda,\theta)$ will return the tendencies for $q_{Cl}$ (mixing ratio) and $q_{Cl2}$ (mixing ratio), are provided as supplemental material in \cite{lauritzen2015terminator}. Similarly for the forcing terms.

The `toy' terminator chemistry test here uses the baroclinic wave initialization (described in section \ref{sec:baroclinic_wave}) based on the moist setup and the flow will transport the two species that interact non-linearly with each other through the toy chemistry. A physics time-step of 30 minutes and 15 minutes, respectively, is used. Note that if the user has the baroclinic wave setup the only additional work is to initialize two tracers and implement the chemistry. Example solutions are shown for HOMME (High-Order Method Modeling Environment) \cite{dennis2012camse} spectral elements and HOMME-CSLAM \cite{lauritzen2016camsecslam}. The latter model is based on CSLAM \cite{lauritzen2010conservative} transport of tracers consistently coupled with spectral element dynamics.

\subsubsection{Diagnostics}
If the initial conditions for $q_{Cl}$ and $2q_{Cl2}$ add up to a constant (as is the case in this setup) then no matter how the individual species evolve the weighted sum $q_{Cly}=q_{Cl}+2q_{Cl2}$ should be constant in space and time. Hence the analytical solution for $q_{Cly}$ is known. The terminator chemistry preserves the linear relationship between $q_{Cl}$ and $q_{Cl2}$ so the only causes for this relationship to break are:
\begin{itemize}
\item the transport operator (usually the limiter/filter) does not exactly preserve linear relations, and/or,
\item physics-dynamics coupling breaks the relationship (see, e.g., Figure \ref{fig:terminator-diags}).
\end{itemize}
The following diagnostics are used in this test case:
\begin{itemize}
\item Average column integrated mixing ratio (two-dimensional variable):
\begin{equation}
<q>=\frac{\int_{z=0}^{z_{top}} q\, dz}{\int_{z=0}^{z_{top}} dz}.
\end{equation}
where $q=q_{Cl},q_{Cl2}$. The global integrals should be computed consistently with the numerical method (preferably `inline' in the source code on the native grid and not using interpolated data).
\item $\ell_2(t)$, $\ell_\infty(t)$ and relative mass change $\Delta M(t)$ error norms for Cl$_y$:
\begin{align}
\ell_2(t)&=\frac{\sqrt{\int_{z=0}^{z_{top}}\left(<q_{Cly}>-4.0\times 10^{-6}\right)^2 dz}}{\sqrt{\int_{z=0}^{z_{top}}\left(4.0\times 10^{-6}\right)^2 dz}},\\
\ell_\infty(t)&=\frac{\max_{\text{all }\lambda, \theta} |<q_{Cly}>-4.0\times 10^{-6}|}{4.0\times 10^{-6}},\\
\Delta M(t)&=\frac{\int_{z=0}^{z_{top}}q_{Cly}dz-M_0}{M_0}
\end{align}
respectively, where
\begin{equation}
<q_{Cly}>=<q_{Cl}+2q_{Cl2}>.
\end{equation}
and $M_0$ is the initial mass of Cl$_y$
\begin{equation}
M_0=\int_{z=0}^{z_{top}}4.0\times 10^{-6}dz.
\end{equation}
\end{itemize}
Note that if the physics-dynamics coupling procedure breaks $Cl_y$ conservation it can be very subtle but shows in the $\Delta M(t)$ diagnostic (see, e.g., Figure \ref{fig:terminator-diags}).
%\begin{equation}
%q_{Cly}=q_{Cl}+2q_{Cl2}.
%\end{equation}
%The exact solution is that $q_{Cly}$ remains constant in time:
%\begin{equation}
%q_{Cly}(t)=q_{Cly}(t=0)=4.0\times 10^{-6}\, kg/kg.
%\end{equation}






\begin{figure}[tb]
\center\includegraphics[width=\linewidth]{terminator_diags-t_phys900.pdf}
\center\includegraphics[width=\linewidth]{terminator_diags-t_phys1800.pdf}
  \caption{Global error norms $\ell_2(t)$ (column 1), $\ell_\infty(t)$ (column 2) and relative mass change $\Delta M(t)$ (column 3) for $q_{Cly}$ for HOMME  (row 1 and 3) and HOMME-CSLAM  (row 2 and 4) with physics/chemistry time-step of 15 minutes (row 1 and 2) and 30 minutes (row 3 and 4), respectively. The non-conservation of Cl$_y$ mass ($\Delta M(t)$) for physics time-step of 30 minutes is due to physics-dynamics coupling in which the tendencies are altered if they result in negative mixing ratios. For HOMME-CSLAM this happens at one point around day 16 and 18, respectively, whereas it happens frequently (in time and space) for HOMME.}\label{fig:terminator-diags}
\end{figure} 



%\cite{LCLVT2015GMD}
%\subsection{Grid spacings, simulation time, output and diagnostics}
%
%\begin{itemize}
%\item Dry simulations with the Toy Chemistry module (Appendix \ref{sec:ToyChemistry}) should be performed at 1$^\circ$ resolution with 30 vertical levels for 12 days.
%\item Plots of $q_{Cl}$, $q_{Cl2}$ and $q_T$ at 5000m altitude should be produced after $1$, $5$ and $12$ days. 
%\item Experiments could address the coupling frequency between the dynamics and physics.
%\end{itemize}




\subsection{Grid spacings, simulation time, output and diagnostics}

Reference simulations (dry and moist) are performed at 1$^\circ$ resolution with 30 vertical levels for 15 days.  Plots should be produced for the moist simulation and the anomaly between moist and dry simulations at day 9, 12 and 15.

\begin{itemize}
\item Plots of minimum surface pressure over the duration of the simulation for both dry and moist configurations.
\item A variable resolution simulation should be performed that (a) studies the effect of the baroclinic wave transitioning from coarse resolution to fine resolution and (b) studies the effect of enhanced resolution near the front.
\item Terminator chemistry (use physics time-step of 30 minutes and 15 minutes): 
\begin{itemize}
\item Please plot contour lines for $<q_{Cl}>/(4.0\times 10^{-6})$ at day 9. Contour interval must be 0.1 with zero contour. Please offset the zero contour \verb+-1.0E-12+ to avoid contouring round-off undershoots. Contour levels used in Figure \ref{fig:terminator-2d-se} are\\
(\verb+-0.1,-1.0E-12,0.1,0.2,0.3,0.4,0.5,0.6,0.7,0.8,0.9,1.0,1.2+)
\item Please plot contour lines for $<q_{Cl2}>/(4.0\times 10^{-6})$ at day 9. Contour interval must be 0.04 with round-off offset zero contour.  Contour levels used in Figure \ref{fig:terminator-2d-se} are\\
(\verb+-0.04,-1.0E-12,0.04,0.08,0.12,0.16,0.20,0.24,0.28,+\\
\verb+0.32,0.36,0.40,0.44,0.48+)
\item Please plot contour lines for $<q_{Cly}>/(4.0\times 10^{-6})$ at day 9. Contour interval must be 0.04 (or 0.02 or 0.01 depending on your data) excluding the 1.0 contour but symmetric about 1.0. Contour levels used in Figure \ref{fig:terminator-2d-se} are\\
(\verb+0.78,0.82,0.86,0.90,0.94,0.98,1.02,1.06,1.1,1.14,1.18,1.22,1.24,+
\verb+1.08,1.12,1.16,1.20+)
\item Please plot global error norms $\ell_2(t)$, $\ell_\infty(t)$ and relative mass change $\Delta M(t)$ for $q_{Cly}$ as a function of time from day 0 to 30 preferably with 3 hourly time spacing. Use vertical axis adjusted to your data. See example on Figure \ref{fig:terminator-diags}.
\end{itemize}
\end{itemize}

\begin{figure}[t]
\includegraphics[width=0.9\linewidth]{terminator-2d-se-day9.pdf}
\includegraphics[width=0.9\linewidth]{terminator-2d-cslam-day9.pdf}
\caption{Upper panel of 4 plots is contour lines of $<q_{Cl}>/(4.0\times 10^{-6})$, (upper left), $<q_{Cl2}>/(4.0\times 10^{-6})$ (upper right), $<q_{Cly}>$ (lower left) and surface pressure (lower right) at day 9 simulated with HOMME at approximately $1^\circ$ resolution (30x30 elements on each cubed-sphere panel with 4x4 quadrature points in each element). Lower panel of 4 plots is the same as the upper panel but simulated with HOMME-CSLAM with 3x3 control volumes within each element. The results are based on the moist baroclinic wave setup (with no moist processes) and a physics-coupling time-step of 15 minutes.}
\label{fig:terminator-2d-se}
\end{figure}

\conclusions  %% \conclusions[modified heading if necessary]
TEXT




\appendix

\section{`Toy' Chemistry} \label{sec:ToyChemistry}

The toy chemistry module represents a simple photolysis-driven chemical reaction that incorporates combination and the dissociation of a chemical species:
\begin{align}
Cl + Cl &= Cl_2 && \mbox{(reaction rate $k_1$)}\\
Cl_2&=Cl+Cl && \mbox{(reaction rate $k_2$)}.
\end{align}

Observe that the total number of molecules of the chemical species are conserved in this reaction,
\begin{equation}
Cl_T=Cl+2 Cl_2.
\end{equation}  Representing the mixing ratios of these species as $q_{Cl}$ and $q_{Cl2}$, we can define the total mixing ratio of Chlorine atoms as
\begin{equation}
q_{Cly} = q_{Cl} + 2 q_{Cl2}.
\end{equation}

The differential equations describing the evolution of $Cl$ and $Cl_2$ under this reaction take the form
\begin{align}
\frac{Dq_{Cl}}{Dt} &= 2k_1 q_{Cl2} -2k_2 q_{Cl}^2, \\
\frac{Dq_{Cl2}}{Dt} &= -k_1 q_{Cl2} + k_2 q_{Cl}^2,
\end{align} where $D/Dt$ denotes the Lagrangian derivative.  Observe that the total mixing ratio of Chlorine atoms then satisfies
\begin{equation}
\frac{Dq_{Cly}}{Dt} = \frac{Dq_{Cl}}{Dt} + 2 \frac{Dq_{Cl2}}{Dt} = 0,
\end{equation} and so the total mixing ratio of Chlorine is held constant.

The two reaction rate coefficient  $k_1$ and $k_2$, representing the the photolytic dissociation and recombination of Chlorine gas are defined as
\begin{align}
k_1(\lambda,\theta)&= \mbox{max} \left[ 0,\sin\theta\sin\theta_c+\cos\theta\cos\theta_c, \label{app:term_k1}
\cos(\lambda-\lambda_c) \right] \\
k_2(\lambda,\theta)&=1,\label{app:term_k2}
\end{align} where $(\lambda_c, \theta_c)=(20^\circ N, 300^\circ E)$ denote the sub-solar point on the Earth's surface.

%\subsection{}                               %% Appendix A1, A2, etc.


\authorcontribution{TEXT}

\begin{acknowledgements}
DCMIP2016 is sponsored by the National Center for Atmospheric Research Computational Information Systems Laboratory, the Department of Energy Office of Science (award no. DE-SC0016015), the National Science Foundation (award no. 1629819), the National Aeronautics and Space Administration (award no. {\color{red}??}), the National Oceanic and Atmospheric Administration (award no. {\color{red}??}), the Office of Naval Research and CU Boulder Research Computing.  This work was made possible with support from our student and postdoctoral participants: Sabina Abba Omar, Scott Bachman, Amanda Back, Tobias Bauer, Vinicius Capistrano, Spencer Clark, Ross Dixon, Christopher Eldred, Robert Fajber, Jared Ferguson, Emily Foshee, Ariane Frassoni, Alexander Goldstein, Jorge Guerra, Chasity Henson, Adam Herrington, Tsung-Lin Hsieh, Dave Lee, Theodore Letcher, Weiwei Li, Laura Mazzaro, Maximo Menchaca, Jonathan Meyer, Farshid Nazari, John O'Brien, Bjarke Tobias Olsen, Hossein Parishani, Charles Pelletier, Thomas Rackow, Kabir Rasouli, Cameron Rencurrel, Koichi Sakaguchi, G\"okhan Sever, James Shaw, Konrad Simon, Abhishekh Srivastava, Nicholas Szapiro, Kazushi Takemura, Pushp Raj Tiwari, Chii-Yun Tsai, Richard Urata, Karin van der Wiel, Lei Wang, Eric Wolf, Zheng Wu, Haiyang Yu, Sungduk Yu and Jiawei Zhuang.  We would also like to thank Rich Loft, Cecilia Banner, Kathryn Peczkowicz and Rory Kelly (NCAR), Carmen Ho, Perla Dinger, and Gina Skyberg (UC Davis) and Kristi Hansen (University of Michigan) for administrative support during the workshop and summer school.  
\end{acknowledgements}



%% REFERENCES

%% Since the Copernicus LaTeX package includes the BibTeX style file copernicus.bst,
%% authors experienced with BibTeX only have to include the following two lines:
%%
\bibliographystyle{copernicus}
\bibliography{DCMIP2016-Part2}
%%
%% URLs and DOIs can be entered in your BibTeX file as:
%%
%% URL = {http://www.xyz.org/~jones/idx_g.htm}
%% DOI = {10.5194/xyz}


%% LITERATURE CITATIONS
%%
%% command                        & example result
%% \citet{jones90}|               & Jones et al. (1990)
%% \citep{jones90}|               & (Jones et al., 1990)
%% \citep{jones90,jones93}|       & (Jones et al., 1990, 1993)
%% \citep[p.~32]{jones90}|        & (Jones et al., 1990, p.~32)
%% \citep[e.g.,][]{jones90}|      & (e.g., Jones et al., 1990)
%% \citep[e.g.,][p.~32]{jones90}| & (e.g., Jones et al., 1990, p.~32)
%% \citeauthor{jones90}|          & Jones et al.
%% \citeyear{jones90}|            & 1990



%% FIGURES

%% ONE-COLUMN FIGURES

%%f
%\begin{figure}[t]
%\includegraphics[width=8.3cm]{FILE NAME}
%\caption{TEXT}
%\end{figure}
%
%%% TWO-COLUMN FIGURES
%
%%f
%\begin{figure*}[t]
%\includegraphics[width=12cm]{FILE NAME}
%\caption{TEXT}
%\end{figure*}
%
%
%%% TABLES
%%%
%%% The different columns must be seperated with a & command and should
%%% end with \\ to identify the column brake.
%
%%% ONE-COLUMN TABLE
%
%%t
%\begin{table}[t]
%\caption{TEXT}
%\begin{tabular}{column = lcr}
%\tophline
%
%\middlehline
%
%\bottomhline
%\end{tabular}
%\belowtable{} % Table Footnotes
%\end{table}
%
%%% TWO-COLUMN TABLE
%
%%t
%\begin{table*}[t]
%\caption{TEXT}
%\begin{tabular}{column = lcr}
%\tophline
%
%\middlehline
%
%\bottomhline
%\end{tabular}
%\belowtable{} % Table Footnotes
%\end{table*}
%
%
%%% NUMBERING OF FIGURES AND TABLES
%%%
%%% If figures and tables must be numbered 1a, 1b, etc. the following command
%%% should be inserted before the begin{} command.
%
%\addtocounter{figure}{-1}\renewcommand{\thefigure}{\arabic{figure}a}
%
%
%%% MATHEMATICAL EXPRESSIONS
%
%%% All papers typeset by Copernicus Publications follow the math typesetting regulations
%%% given by the IUPAC Green Book (IUPAC: Quantities, Units and Symbols in Physical Chemistry,
%%% 2nd Edn., Blackwell Science, available at: http://old.iupac.org/publications/books/gbook/green_book_2ed.pdf, 1993).
%%%
%%% Physical quantities/variables are typeset in italic font (t for time, T for Temperature)
%%% Indices which are not defined are typeset in italic font (x, y, z, a, b, c)
%%% Items/objects which are defined are typeset in roman font (Car A, Car B)
%%% Descriptions/specifications which are defined by itself are typeset in roman font (abs, rel, ref, tot, net, ice)
%%% Abbreviations from 2 letters are typeset in roman font (RH, LAI)
%%% Vectors are identified in bold italic font using \vec{x}
%%% Matrices are identified in bold roman font
%%% Multiplication signs are typeset using the LaTeX commands \times (for vector products, grids, and exponential notations) or \cdot
%%% The character * should not be applied as mutliplication sign
%
%
%%% EQUATIONS
%
%%% Single-row equation
%
%\begin{equation}
%
%\end{equation}
%
%%% Multiline equation
%
%\begin{align}
%& 3 + 5 = 8\\
%& 3 + 5 = 8\\
%& 3 + 5 = 8
%\end{align}
%
%
%%% MATRICES
%
%\begin{matrix}
%x & y & z\\
%x & y & z\\
%x & y & z\\
%\end{matrix}
%
%
%%% ALGORITHM
%
%\begin{algorithm}
%\caption{�}
%\label{a1}
%\begin{algorithmic}
%�
%\end{algorithmic}
%\end{algorithm}
%
%
%%% CHEMICAL FORMULAS AND REACTIONS
%
%%% For formulas embedded in the text, please use \chem{}
%
%%% The reaction environment creates labels including the letter R, i.e. (R1), (R2), etc.
%
%\begin{reaction}
%%% \rightarrow should be used for normal (one-way) chemical reactions
%%% \rightleftharpoons should be used for equilibria
%%% \leftrightarrow should be used for resonance structures
%\end{reaction}
%
%
%%% PHYSICAL UNITS
%%%
%%% Please use \unit{} and apply the exponential notation


\end{document}
